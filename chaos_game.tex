% Options for packages loaded elsewhere
\PassOptionsToPackage{unicode}{hyperref}
\PassOptionsToPackage{hyphens}{url}
%
\documentclass[
]{article}
\title{Chaos Game}
\author{}
\date{\vspace{-2.5em}}

\usepackage{amsmath,amssymb}
\usepackage{lmodern}
\usepackage{iftex}
\ifPDFTeX
  \usepackage[T1]{fontenc}
  \usepackage[utf8]{inputenc}
  \usepackage{textcomp} % provide euro and other symbols
\else % if luatex or xetex
  \usepackage{unicode-math}
  \defaultfontfeatures{Scale=MatchLowercase}
  \defaultfontfeatures[\rmfamily]{Ligatures=TeX,Scale=1}
\fi
% Use upquote if available, for straight quotes in verbatim environments
\IfFileExists{upquote.sty}{\usepackage{upquote}}{}
\IfFileExists{microtype.sty}{% use microtype if available
  \usepackage[]{microtype}
  \UseMicrotypeSet[protrusion]{basicmath} % disable protrusion for tt fonts
}{}
\makeatletter
\@ifundefined{KOMAClassName}{% if non-KOMA class
  \IfFileExists{parskip.sty}{%
    \usepackage{parskip}
  }{% else
    \setlength{\parindent}{0pt}
    \setlength{\parskip}{6pt plus 2pt minus 1pt}}
}{% if KOMA class
  \KOMAoptions{parskip=half}}
\makeatother
\usepackage{xcolor}
\IfFileExists{xurl.sty}{\usepackage{xurl}}{} % add URL line breaks if available
\IfFileExists{bookmark.sty}{\usepackage{bookmark}}{\usepackage{hyperref}}
\hypersetup{
  pdftitle={Chaos Game},
  hidelinks,
  pdfcreator={LaTeX via pandoc}}
\urlstyle{same} % disable monospaced font for URLs
\usepackage[margin=1in]{geometry}
\usepackage{graphicx}
\makeatletter
\def\maxwidth{\ifdim\Gin@nat@width>\linewidth\linewidth\else\Gin@nat@width\fi}
\def\maxheight{\ifdim\Gin@nat@height>\textheight\textheight\else\Gin@nat@height\fi}
\makeatother
% Scale images if necessary, so that they will not overflow the page
% margins by default, and it is still possible to overwrite the defaults
% using explicit options in \includegraphics[width, height, ...]{}
\setkeys{Gin}{width=\maxwidth,height=\maxheight,keepaspectratio}
% Set default figure placement to htbp
\makeatletter
\def\fps@figure{htbp}
\makeatother
\setlength{\emergencystretch}{3em} % prevent overfull lines
\providecommand{\tightlist}{%
  \setlength{\itemsep}{0pt}\setlength{\parskip}{0pt}}
\setcounter{secnumdepth}{-\maxdimen} % remove section numbering
\ifLuaTeX
  \usepackage{selnolig}  % disable illegal ligatures
\fi

\begin{document}
\maketitle

Ianuarie, 2022

Realizat de Iordache Ovidiu si Daramus Claudiu-Lucian

\hypertarget{despre-chaos-game}{%
\subsection{Despre Chaos Game}\label{despre-chaos-game}}

\href{https://en.wikipedia.org/wiki/Chaos_game}{Wikipedia}

În matematică, termenul de Chaos Game se referea inițial la o metodă de
creare a unui fractal, folosind un poligon și un punct inițial selectat
la întâmplare în interiorul acestuia. Fractalul este creat prin crearea
iterativă a unei secvențe de puncte, începând cu punctul aleator
inițial, în care fiecare punct din șir este o fracțiune dată din
distanța dintre punctul anterior și unul dintre vârfurile poligonului;
vârful este ales la întâmplare în fiecare iterație. Repetarea acestui
proces iterativ de un număr mare de ori, selectând vârful la întâmplare
la fiecare iterație și aruncând primele câteva puncte din secvență, va
produce adesea (dar nu întotdeauna) o formă fractală. Folosind un
triunghi obișnuit și factorul 1/2 va rezulta triunghiul Sierpinski, în
timp ce crearea aranjamentului adecvat cu patru puncte și un factor 1/2
va crea o afișare a unui „Tetraedru Sierpinski'', analogul
tridimensional al Sierpinski. triunghi. Pe măsură ce numărul de puncte
este crescut la un număr N, aranjamentul formează un (N-1)-dimensional
Sierpinski Simplex corespunzător.

Termenul a fost generalizat pentru a se referi la o metodă de generare a
atractorului, sau a punctului fix, al oricărui sistem de funcții iterate
(IFS). Începând cu orice punct \(x_0\), iterațiile succesive sunt
formate ca \(x_k+1\) = \(f_r\)(\(x_k\)), unde fr este un membru al
IFS-ului dat selectat aleatoriu pentru fiecare iterație. Iterațiile
converg către punctul fix al IFS. Ori de câte ori \(x_0\) aparține
atractorului IFS, toate iterațiile \(x_k\) rămân în interiorul
atractorului și, cu probabilitatea 1, formează o mulțime densă în acesta
din urmă.

Metoda „chaos game'' trasează punctele în ordine aleatorie peste tot
atractorul. Acest lucru este în contrast cu alte metode de desenare a
fractalilor, care testează fiecare pixel de pe ecran pentru a vedea dacă
aparține fractalului. Forma generală a unui fractal poate fi trasată
rapid cu metoda „chaos game'', dar poate fi dificil să trasezi unele
zone ale fractalului în detaliu.

Cu ajutorul „chaos game'' se poate realiza un nou fractal și în timpul
realizării noului fractal pot fi obținuți câțiva parametri. Acești
parametri sunt utili pentru aplicații ale teoriei fractale, cum ar fi
clasificarea și identificarea. Noul fractal este auto-similar cu
originalul în unele caracteristici importante, cum ar fi dimensiunea
fractală.

Dacă în „chaos game'' începeți de la fiecare vârf și treceți prin toate
căile posibile pe care le poate parcurge jocul, veți obține aceeași
imagine ca și când luați o singură cale aleatorie. Cu toate acestea,
luarea a mai mult de o cale se face rar, deoarece suprasolicitarea
pentru urmărirea fiecărei căi o face mult mai lent de calculat. Această
metodă are avantajele de a ilustra modul în care fractalul este format
mai clar decât metoda standard, precum și de a fi deterministă.

\hypertarget{forme-fractali}{%
\subsubsection{Forme Fractali:}\label{forme-fractali}}

Folosind un triunghi obișnuit și factorul 1/2 va rezulta
\href{https://en.wikipedia.org/wiki/Sierpi\%C5\%84ski_triangle}{Triunghiul
Sierpiński}:
\includegraphics{https://upload.wikimedia.org/wikipedia/commons/f/ff/Sierpinski_chaos_animated.gif}

Un pentaflake, sau
\href{https://en.wikipedia.org/wiki/N-flake\#Pentaflake}{Pentagon
Sierpiński}, este format din fulgi succesivi din șase pentagoane
regulate. Fiecare fulg este format prin plasarea unui pentagon în
fiecare colț și unul în centru. golden ration este de
\textasciitilde0,63:
\includegraphics{https://upload.wikimedia.org/wikipedia/commons/c/c7/V4jump2_5_center.gif}

\end{document}
